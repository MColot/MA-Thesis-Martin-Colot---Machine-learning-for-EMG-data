\documentclass{article}
\usepackage[utf8]{inputenc}
\usepackage[margin=1in]{geometry}
\usepackage{mathtools}
\DeclarePairedDelimiter{\ceil}{\lceil}{\rceil}
\usepackage{amsfonts} 
\usepackage{caption}
\usepackage{subcaption}
\usepackage{float}

\title{Master Thesis\\Machine Learning For EMG Data}
\author{Martin Colot}
\date{2021}

\begin{document}

\maketitle


\section{Introduction}


\section{State of the art}

\subsection{EMG}

\begin{itemize}
    \item What is an EMG signal
    \item EMG and EEG
    \item EMG and ENG \\
    https://pubmed.ncbi.nlm.nih.gov/33091891/  \\
    https://pubmed.ncbi.nlm.nih.gov/29498358/
    \item sEMG and iEMG sensor
    \item high density EMG \\
    https://www.sciencedirect.com/science/article/abs/pii/S1746809419302186 \\
    https://pubmed.ncbi.nlm.nih.gov/22180516/
\end{itemize}


\subsection{Myoelectric hand prosthesis}

\begin{itemize}
    \item Purpose
    \item Existing brands \\
    https://ieeexplore.ieee.org/document/8733629 \\
    https://app.dimensions.ai/details/publication/pub.1112252996?and\_facet\_journal=jour.1041772
    \item Difficulties
    \begin{itemize}
        \item limitation of non-invasive sensor
        \item Lack of EMG data for amputees
        \item Mirrored billateral training \\
        https://pubmed.ncbi.nlm.nih.gov/22180516/ \\
        https://pubmed.ncbi.nlm.nih.gov/22006428/
    \end{itemize}
\end{itemize}


\subsection{Hand gesture prediction}

\begin{itemize}
    \item Applications (prosthetic, VR)
    \item Classification and regression
\end{itemize}

\subsubsection{Preprocessing and feature selection}

\subsubsection{Gesture classification}

\begin{itemize}
    \item Classification techniques \\
    https://journals.physiology.org/doi/pdf/10.1152/jn.00555.2014 \\
    https://www.nature.com/articles/s41551-016-0025
    \item Limitations
\end{itemize}


\subsubsection{Movement regression (joint angle classification)}

\begin{itemize}
    \item Needed for a more natural feeling
    \item 27 degrees of freedom of the hand
    \item Regression techniques \\
    https://jneuroengrehab.biomedcentral.com/articles/10.1186/1743-0003-11-122 \\
    https://www.hindawi.com/journals/isrn/2012/604314/ \\
    https://pubmed.ncbi.nlm.nih.gov/22180516/
\end{itemize}



\subsection{Existing data set of synchronized EMG and hand gesture data}

\begin{itemize}
    \item Which ones exist \\
    https://www.nature.com/articles/s41597-019-0285-1 \\
    https://www.nature.com/articles/sdata201453 \\
    https://jneuroengrehab.biomedcentral.com/articles/10.1186/1743-0003-11-122
    \item Data collection
    \begin{itemize}
        \item Experimental setup
        \begin{itemize}
            \item EMG sensor
            \item Hand pose Estimation
        \end{itemize}
        \item Electrodes placement
        \begin{itemize}
            \item Palpation\\
            https://journals.plos.org/plosone/article?id=10.1371/journal.pone.0186132\\
            https://www.hindawi.com/journals/isrn/2012/604314/
            \item Identified zones of activity \\
            https://jneuroengrehab.biomedcentral.com/articles/10.1186/s12984-018-0437-0
            \item Arm band \\
            “CAMERA - GUIDED INTERPRETATION OF ( 56 ) NEUROMUSCULAR SIGNALS” from Facebook \\
            https://www.mdpi.com/2079-9292/9/12/2143/pdf \\
            https://www.mdpi.com/1424-8220/19/14/3170/pdf-vor
        \end{itemize}
        \item Gesture performed
        \begin{itemize}
            \item Single finger motions \\
            https://www.researchgate.net/publication/341629918\_Simultaneous\_Hand\_Gesture\_Classification\_and\_Finger\_Angle\_Estimation\_via\_a\_Novel\_Dual-Output\_Deep\_Learning\_Model
            
            \item Activities of daily living (ADL) \\
            https://www.tandfonline.com/doi/abs/10.3109/02844319509034334
            
            \item Sign language \\
            https://www.mdpi.com/1424-8220/19/14/3170/pdf-vor \\
            https://www.mdpi.com/1424-8220/20/10/2972 \\
            https://www.researchgate.net/publication/243769865\_A\_Sign\_Language\_Recognition\_System\_Using\_Hidden\_Markov\_Model\_and\_Context\_Sensitive\_Search
            \item Irregular moves
            \item Maximum voluntary contraction (MVC) \\
            https://pubmed.ncbi.nlm.nih.gov/29355119/ 
        \end{itemize}
    \end{itemize}
    \item Hand position data representation \\
    https://www.sciencedirect.com/science/article/abs/pii/S002192900400301X?via\%3Dihub
    \item Synchronization \\
    https://www.researchgate.net/post/How\_can\_I\_synchronize\_EMG\_and\_acceleration\_data 
\end{itemize}

\subsection{Hand tracking using Oculus Quest}

\subsubsection{Motion tracking systems}



\begin{itemize}
    \item CyberGlove
    \item 3D motion tracking cameras
    \item Oculus Quest: Unity and The OVR library \\
    How Oculus uses AI for hand tracking : https://augmentedstartups.medium.com/how-oculus-uses-ai-for-hand-tracking-8d9eb8046029 \\
    Using deep neural networks for accurate hand-tracking on Oculus Quest : https://ai.facebook.com/blog/hand-tracking-deep-neural-networks \\
    DeepHandsVR: Hand Interface Using Deep Learning in Immersive Virtual Reality: https://www.mdpi.com/2079-9292/9/11/1863 \\
    Handcrated and Deep Trackers: Recent Visual Object Tracking Approaches and Trends: https://arxiv.org/pdf/1812.07368.pdf \\
    Hand pose estimation : https://www.tensorflow.org/lite/models/pose\_estimation/overview
    How Accurate is Oculus Quest 2 Hand-tracking Feature?: https://www.youtube.com/watch?v=g8fGShHy3MA&ab\_channel=SpookyFairy  \\
    Hand Physics Lab: Hand Tracking Demos in Oculus Quest!: https://www.youtube.com/watch?v=J0KhC1GpLSQ&ab\_channel=AdamSavage\%E2\%80\%99sTested 
    Hand physic lab: https://sidequestvr.com/app/750/hand-physics-lab \\
    https://developer.oculus.com/documentation/unity/unity-utilities-overview/
\end{itemize}


\section{Realisation: Creation of a data set}

\subsection{Experimental setup}

\subsection{Hand tracking collection}

\begin{itemize}
    \item Data collection
    \item Interface
\end{itemize}


\subsection{Synchronization of EMG signals}

\begin{itemize}
    \item remote control
    \begin{itemize}
        \item UTC
        \item Trigger signal
    \end{itemize}
\end{itemize}


\subsection{Data collection protocol}
https://github.com/MColot/MA-Thesis-Martin-Colot---Machine-learning-for-EMG-data/blob/master/notes/dataAcquisitionProtocol.pdf

\begin{itemize}
    \item Chosen electrodes locations
    \item Chosen Gestures
\end{itemize}

\section{Conclusion and future work}



\end{document}
