\documentclass[12pt]{article}
\usepackage[utf8]{inputenc}
\usepackage[margin=2.7cm]{geometry}
\usepackage{mathtools}
\DeclarePairedDelimiter{\ceil}{\lceil}{\rceil}
\usepackage{amsfonts} 
\usepackage{caption}
\usepackage{subcaption}
\usepackage{float}
\usepackage{titlesec}
\setcounter{secnumdepth}{5}
\usepackage{url}
\usepackage{xcolor}
\usepackage{subfiles}

\renewcommand{\baselinestretch}{1.2}


\title{Master Thesis\\Machine Learning For EMG Data}
\author{Martin Colot}
\date{2021}

\begin{document}
	
	\maketitle
	
	\tableofcontents
	\newpage
	
	\part{Introduction}
	
	
	\newpage
	\part{State of the art}
	
	
	
	\section{EMG}
	
	\subsection{What is an EMG signal}
	
	\subsection{EMG, ENG and EEG}
	https://pubmed.ncbi.nlm.nih.gov/33091891/  \\
		https://pubmed.ncbi.nlm.nih.gov/29498358/
		
	\subsection{sEMG and iEMG sensor}
	
	\subsection{high density EMG}
	https://www.sciencedirect.com/science/article/abs/pii/S1746809419302186 \\
	https://pubmed.ncbi.nlm.nih.gov/22180516/

	
	
	\section{Myoelectric hand prosthesis}
	
	Human-computer interface (HCI)
	
	
	\subsection{Existing brands}
	https://ieeexplore.ieee.org/document/8733629 \\
	https://app.dimensions.ai/details/publication/pub.1112252996?and\_facet\_journal=jour.1041772
		
	\subsection{Difficulties}
	\begin{itemize}
		\item limitation of non-invasive sensor
		\item Lack of EMG data for amputees
		\item Mirrored billateral training \\
		https://pubmed.ncbi.nlm.nih.gov/22180516/ \\
		https://pubmed.ncbi.nlm.nih.gov/22006428/
	\end{itemize}
	
	
	
	\section{Hand gesture estimation}
	
	\subfile{StateOfTheArt_poseEstimation}
	
	
	
	
	\section{Existing data set of synchronized EMG and hand gesture data}
	
	\subfile{stateOfTheArt_DataCollection_existingDatasets}
	
	
	
	\subsection{Motion capture technologies \label{sec:motionCaptureTech}}
	
	\subfile{stateOfTheArt_DataCollection_motionCapture}
	
	
	
	
	\subsection{sEMG Electrodes placement}
	
	\subfile{stateOfTheArt_DataCollection_electrodesPlacement}
	
	
	
	
	
	\subsection{Gestures performed}
	
	\subfile{stateOfTheArt_DataCollection_Gesture}
	
	
	\subsection{Hand posture data representation}
	
	Each hand motion tracking device has its own representation of the joint angles of the hand. The KIN-MUS UJI data set \cite{ref:KinMusUji} has chosen to follow the international Society of Biomechanics (ISB) sign criteria for representation of the anatomical angles of the hand and the wrist \cite{ref:handAnatomicalAngles}. The interest in such a standardisation is that it allows teams of researchers to use the data set without depending on the capture system.
	
	
	
	\subsubsection{Gesture duration and repetition}
	
	The referenced studies on creation of a data set enable to do predictions (regression or classification) of the hand kinematics based on sEMG signal are using a very low number of repetitions (typically, less than 10 repetition of a single gesture) with a pose duration of a few seconds. Those that focus on gesture classification have a long time of pose duration because they need an external actor to manually mark the beginning and the ending of each pose. Ultimately, there is no benchmark on the duration and number of repetitions needed to get a good estimation accuracy using state of the art machine learning algorithms.
	
	
	
	\subsection{Synchronization}
	
	The implementation of the synchronization software of the referenced data sets is always kept secret. It is nevertheless highly dependant on the hardware used for data collection. We can however cite some techniques of synchronizations that could be used.
	
	\begin{itemize}
		\item Synchronization of the computers timestamps (via UTC)
		\item Sacrifice of one sEMG electrode to use as a trigger signal (\url{https://www.researchgate.net/post/How\_can\_I\_synchronize\_EMG\_and\_acceleration\_data })
		\item Some systems of sEMG sensors have a specific entry used to send trigger signals that start/stop the recording
	\end{itemize}

	In the end, all captation device also has a fixed record delay that must be taken into account when synchronizing the different data. This delay is always given in the hardware official documentation. As an example, the Cometa EMG sensor has a collection delay of 14ms.
	
	
	
	
	\section{Motion tracking using the Oculus Quest}
	
	
	\subfile{stateOfTheArt_DataCollection_OculusQuest}
	
	
	\newpage
	\part{Realisation: Creation of a synchronized EMG/EEG/hand-gesture data set}
	
	\subfile{realisations_Devlopement}
	
	
	
	
	\section{Data collection protocol}
	
	\subsection{Placement of the sEMG electrodes}
	
	\subsection{Gestures to perform}
	
	
	\newpage
	\part{Conclusion and future work}
	
	\newpage
	\bibliographystyle{plain}
	\bibliography{biblio}
	\nocite{*}
	
	
\end{document}
